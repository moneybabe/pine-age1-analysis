\documentclass{article}
\usepackage{amsfonts, amsmath, amssymb, amsthm} % Math notations imported
\usepackage{enumitem}
\usepackage[margin=1in]{geometry}
\usepackage{sistyle}
\SIthousandsep{,}

\newtheorem{thm}{Theorem}
\newtheorem{prop}[thm]{Proposition}
\newtheorem{cor}[thm]{Corollary}

% title information
\title{Pine Age 1 Reward Analysis}
\author{Neo Lee}
\date{02/19/2023}

% main content
\begin{document} 

% placing title information; comment out if using fancyhdr
\maketitle 

\section*{Introduction}
This paper aims to analyze a game that is often implemented as a reward system and try to find out the best strategy to maximize the reward.
The rules are as follow:

\begin{enumerate}
    \item 
    There are in total 12 rounds, and at the end of each round, $\num{250000}$ tokens $(T)$ will be distributed to the players.

    \item
    In each round, all players will bet a certain amount of $T$ into a pool that is independent of the $\num{250000}T$ pool. 
    All the betted $T$ by players are unredeemable. 

    \item
    The amount of $T$ that each participant receives at the end of each round is proportional to the amount that they betted in the pool. 
    For example, if player 1 betted $10T$ while others betted $40T$ in total, player 1 will receive $\frac{10}{10+40}=\frac{1}{5}$ of the $\num{250000}T$, which is $\num{50000}T$.

    \item 
    The invested pool is cumulative. For example, if player 1 betted $10T$ and others betted $40T$ in first round, then player 1 betted $20T$ and others betted $20T$ in second round, player 1 will receive $\frac{1}{5}$ of $\num{250000}T$ in first round and $\frac{10+20}{10+40+20+20}=\frac{1}{3}$ of $\num{250000}T$ in second round.

    \item
    The reward that a player receive at the end of $i$-th round can be formulated as follow: 
    \begin{align}
        T_{i,reward} = \frac{\sum_{k=1}^{i}T_{k,betted}}{\sum_{k=1}^{i}T_{k,total}}
    \end{align}
\end{enumerate}

\section*{Tentative Analysis (Aggresive Player)}
We can categorize all players into two groups.
The first group of players start playing since the first round, while the second group do not join the first round.

Since the game is cumulative, it is natural for the players from first group to bet all their $T$ in first round. 





























































\section*{Assumptions (Conservative Player)}
\begin{enumerate}
    \item The game will be analyzed from My perspective as one of the player.
    
    \item The price of $T$ is constant throughout the entire event. 
    Therefore, we will use $T$ as the unit of account for this paper.

    \item I and every player never want a loss by the end of $i$th epoch.
    
    \item I and every player always want to maximize their $T_{reward}$ for each epoch.
    
    \item Let $mb$ be my bet, $tb$ be total bet, $ob$ be others' bet.
\end{enumerate}

\section*{First Epoch}
Assume there are more than one player in this game, then $T_{tb} > T_{mb}$. 
Therefore, to maximize $T_{reward}$, it is natural for me and every player to burn more $T$ to increase $T_{mb}$ thus $\frac{T_{mb}}{T_{tb}}$.

For me to not suffer a loss by the end of the first epoch, it means 
\begin{align}
    T_{reward} = \num{250000} \times \frac{T_{mb}}{T_{tb}} & \ge T_{mb} \\
    \frac{\num{250000}}{T_{tb}} & \ge 1
\end{align}

Since me and every player always wants to burn more $T$, $T_{tb}$ will continue to rise until it converges to $\num{250000}$.
Therefore, we can assume that $T_{1,tb} = \num{250000}$ and $T_{mb} = \num{250000} - T_{ob}$.

\section*{Second Epoch}
To not suffer a loss by the end of the second epoch, it means 
\begin{align}
    \num{250000} \times \frac{T_{1,mb}}{T_{1,tb}} + \num{250000} \times \frac{T_{1,mb} + T_{2,mb}}{T_{1,tb} + T_{2,tb}} & \ge T_{1,mb} + T_{2,mb} \\
    T_{1,mb} + \num{250000} \times \frac{T_{1,mb} + T_{2,mb}}{\num{250000} + T_{2,tb}} & \ge T_{1,mb} + T_{2,mb} \\
    \num{250000} \times \frac{T_{1,mb} + T_{2,mb}}{\num{250000} + T_{2,tb}} & \ge T_{2,mb} \\
    \num{250000} \times T_{1,mb} + \num{250000} \times T_{2,mb} & \ge \num{250000} \times T_{2,mb} + T_{2,mb} \times T_{2,tb} \\
    T_{2,mb} & \le \frac{\num{250000} \times T_{1,mb}}{T_{2,tb}} \\
    T_{2,mb} & \le \frac{\num{250000} \times T_{1,mb}}{T_{2,mb} + T_{ob}} \\
    (T_{2,mb})^2 + T_{2,ob} \times T_{2,mb} -\num{250000} \times T_{1,mb} & \le 0
\end{align}


\end{document}